\documentclass[10pt]{article}\usepackage[]{graphicx}\usepackage[]{color}
%% maxwidth is the original width if it is less than linewidth
%% otherwise use linewidth (to make sure the graphics do not exceed the margin)
\makeatletter
\def\maxwidth{ %
  \ifdim\Gin@nat@width>\linewidth
    \linewidth
  \else
    \Gin@nat@width
  \fi
}
\makeatother

\definecolor{fgcolor}{rgb}{0.345, 0.345, 0.345}
\newcommand{\hlnum}[1]{\textcolor[rgb]{0.686,0.059,0.569}{#1}}%
\newcommand{\hlstr}[1]{\textcolor[rgb]{0.192,0.494,0.8}{#1}}%
\newcommand{\hlcom}[1]{\textcolor[rgb]{0.678,0.584,0.686}{\textit{#1}}}%
\newcommand{\hlopt}[1]{\textcolor[rgb]{0,0,0}{#1}}%
\newcommand{\hlstd}[1]{\textcolor[rgb]{0.345,0.345,0.345}{#1}}%
\newcommand{\hlkwa}[1]{\textcolor[rgb]{0.161,0.373,0.58}{\textbf{#1}}}%
\newcommand{\hlkwb}[1]{\textcolor[rgb]{0.69,0.353,0.396}{#1}}%
\newcommand{\hlkwc}[1]{\textcolor[rgb]{0.333,0.667,0.333}{#1}}%
\newcommand{\hlkwd}[1]{\textcolor[rgb]{0.737,0.353,0.396}{\textbf{#1}}}%
\let\hlipl\hlkwb

\usepackage{framed}
\makeatletter
\newenvironment{kframe}{%
 \def\at@end@of@kframe{}%
 \ifinner\ifhmode%
  \def\at@end@of@kframe{\end{minipage}}%
  \begin{minipage}{\columnwidth}%
 \fi\fi%
 \def\FrameCommand##1{\hskip\@totalleftmargin \hskip-\fboxsep
 \colorbox{shadecolor}{##1}\hskip-\fboxsep
     % There is no \\@totalrightmargin, so:
     \hskip-\linewidth \hskip-\@totalleftmargin \hskip\columnwidth}%
 \MakeFramed {\advance\hsize-\width
   \@totalleftmargin\z@ \linewidth\hsize
   \@setminipage}}%
 {\par\unskip\endMakeFramed%
 \at@end@of@kframe}
\makeatother

\definecolor{shadecolor}{rgb}{.97, .97, .97}
\definecolor{messagecolor}{rgb}{0, 0, 0}
\definecolor{warningcolor}{rgb}{1, 0, 1}
\definecolor{errorcolor}{rgb}{1, 0, 0}
\newenvironment{knitrout}{}{} % an empty environment to be redefined in TeX

\usepackage{alltt}
\usepackage{graphicx, verbatim}
\usepackage{amsmath}
\usepackage{amssymb}
\usepackage{amscd}
\usepackage{lipsum}
\usepackage{blindtext}
\usepackage{todonotes}
\usepackage[tableposition=top]{caption}
\usepackage{ifthen}
\usepackage[utf8]{inputenc}
\usepackage{graphicx}
\usepackage{caption}
\setlength{\textwidth}{6.5in} 
\setlength{\textheight}{9in}
\setlength{\oddsidemargin}{0in} 
\setlength{\evensidemargin}{0in}
\setlength{\topmargin}{-1.5cm}
\setlength{\parindent}{0cm}
\usepackage{setspace}
\usepackage{float}
\usepackage{amssymb}
\usepackage[utf8]{inputenc}
\usepackage{fancyhdr}
\usepackage{tabularx}

\usepackage{hyperref}
\hypersetup{
  colorlinks   = true, %Colours links instead of ugly boxes
  urlcolor     = blue, %Colour for external hyperlinks
  linkcolor    = blue, %Colour of internal links
  citecolor   = red %Colour of citations
}
\usepackage[backend=bibtex ,sorting=none]{biblatex}
%\usepackage[backend=biber ,sorting=none]{biblatex}
\bibliography{references}

\begin{filecontents*}{references.bib}

\end{filecontents*}

%\fancyhf{}
\rfoot{Eyad Elyan \thepage}
\singlespacing
\usepackage[affil-it]{authblk} 
\usepackage{etoolbox}
\usepackage{lmodern}
\IfFileExists{upquote.sty}{\usepackage{upquote}}{}
\begin{document}


\title{\LARGE Coursework  \\ Exemplary Report - CMM507}

\author{Eyad Elyan, \textit{\href{e.elyan@rgu.ac.uk}{e.elyan@rgu.ac.uk}}}
\maketitle
% \begin{flushleft} \today \end{flushleft} 
\noindent\rule{16cm}{0.4pt}
%\underline{\hspace{3cm}
\ \\
%\thispagestyle{empty}

\section*{Objective}

% Edit here to outline the main objectives of your report
You need to study carefully the \textcolor{blue}{.rnw} File, and see how this file was generated. This lab should help you understand the followings:

\begin{itemize}
\item Gain more understanding of reproducible report using \LaTeX and \textcolor{blue}{R}
\item You will learn how to insert chunks of code in your document and adapt its settings (options) to generate graphics using R code 
\item You will learn how to generate tables within your document directly from your excel sheets (interactive reporting)
\item You will learn how to export BibTex items and cite different papers from within your documents 
\item This lab document should help you as a starting template for your coursework
\item Please study carefully the source file and the data files associated with this lab documents (peers, meetings excel sheets)

\end{itemize}

\section{Problem Statement}\label{statement}

% The command below is just to generate some text, edit here for your problem statement
\blindtext[2]

\subsection{Overview}\label{over}

\blindtext[2]

\subsection{Motivation}\label{mot}
%Edit here 
\blindtext[2]

\subsection{Objectives }\label{obj}
%Edit here 
The main objectives of this project can be outlined as follows: 
\section{Research}\label{research}

%Your research goes here, Notice how we cited other people work (check the refernces.bib)
\blindtext[2] Knitr package was used in this work\cite{knitr2013}, more details about clustering can be found at \cite{ELYAN2017220}. A detailed description of class-decomposition can be found at \cite{Elyan2016}. This method was applied to process Engineering Drawings \cite{8489087}.




\section {Methods}\label{methods}

%Edit here 
\blindtext[2]



\subsection{Data Collection}\label{dataset}



%Edit here 
\blindtext[2]


\subsection{Exploration - Pre-processing}\label{explore}

%Edit here 
\blindtext[2]

In this project iris was used, the dataset is made of 150 rows and four features. \\

Notice how we generate graphics within the sweave document. Check the following code, we will create a function that either finds $x^2$ or $x^3$ subject to parameters passed in the function
\begin{knitrout}
\definecolor{shadecolor}{rgb}{0.969, 0.969, 0.969}\color{fgcolor}\begin{kframe}
\begin{alltt}
\hlcom{# create a vector of doubles}
\hlstd{myNumbers} \hlkwb{<-} \hlkwd{seq}\hlstd{(}\hlkwc{from}\hlstd{=}\hlopt{-}\hlnum{1}\hlstd{,}\hlkwc{to}\hlstd{=}\hlnum{1}\hlstd{,}\hlkwc{by}\hlstd{=}\hlnum{.1}\hlstd{)}

\hlcom{# function definition}
\hlstd{toPower} \hlkwb{<-} \hlkwa{function} \hlstd{(}\hlkwc{x}\hlstd{,}\hlkwc{p}\hlstd{=}\hlnum{2}\hlstd{) \{}
    \hlkwa{if} \hlstd{(p}\hlopt{==}\hlnum{2}\hlstd{)}
        \hlkwd{return} \hlstd{(x}\hlopt{*}\hlstd{x)}
    \hlkwa{else if} \hlstd{(p}\hlopt{==}\hlnum{3}\hlstd{)}
        \hlkwd{return} \hlstd{(x}\hlopt{*}\hlstd{x}\hlopt{*}\hlstd{x)}
    \hlkwd{return} \hlstd{(x}\hlopt{*}\hlstd{x)}
\hlstd{\}}

\hlcom{# call function}
\hlstd{squared} \hlkwb{<-} \hlkwd{toPower}\hlstd{(myNumbers)}
\hlstd{cubes} \hlkwb{<-} \hlkwd{toPower}\hlstd{(myNumbers,}\hlnum{3}\hlstd{)}
\end{alltt}
\end{kframe}
\end{knitrout}

An easy way to check that our function is doing the right calculation is to plot the results. The code below will generate a figure similar to Figure~\ref{fig1}: 
\begin{knitrout}
\definecolor{shadecolor}{rgb}{0.969, 0.969, 0.969}\color{fgcolor}\begin{kframe}
\begin{alltt}
\hlkwd{plot}\hlstd{(myNumbers,cubes,}\hlkwc{type}\hlstd{=}\hlstr{'b'}\hlstd{,}\hlkwc{xlab} \hlstd{=} \hlstr{'x'}\hlstd{,} \hlkwc{ylab} \hlstd{=} \hlstr{'x*x'}\hlstd{,}\hlkwc{frame}\hlstd{=}\hlnum{FALSE}\hlstd{,}\hlkwc{col}\hlstd{=}\hlstr{'blue'}\hlstd{)}
\end{alltt}
\end{kframe}
\end{knitrout}


% See carefully how we embed the R code within latex here, check captions, and figure labels 
\begin{figure}[H] %start a figure
\begin{center}

\begin{knitrout}
\definecolor{shadecolor}{rgb}{0.969, 0.969, 0.969}\color{fgcolor}
\includegraphics[width=.47\linewidth]{figure/unnamed-chunk-3-1} 

\end{knitrout}

\caption {Simple Plot of $f(x)=x^3$ Function}
\label{fig1}
\end {center}
\end {figure}




\subsection{Experiments}\label{experiments}

Now we can show how the function $f(x)=x^2$ looks like (Figure~\ref{fig2})

\begin{figure}[H] %start a figure
\begin{center}

\begin{knitrout}
\definecolor{shadecolor}{rgb}{0.969, 0.969, 0.969}\color{fgcolor}
\includegraphics[width=.47\linewidth]{figure/unnamed-chunk-4-1} 

\end{knitrout}

\caption {Simple Plot of $f(x)=x^3$ Function}
\label{fig2}
\end {center}
\end {figure}


\section{Conclusion and Future Work}\label{cdsmote1}

%Edit here 
\blindtext[2]





\section{Project Management}\label{mgt}

\subsection{Project Progress}
%Edit here 
\blindtext[2]
% Pay attention to the code below including the chunk options 
% latex table generated in R 3.3.0 by xtable 1.8-2 package
% Tue Apr 28 23:37:24 2020
\begin{table}[ht]
\centering
\caption{Record of Team Meetings} 
\label{tab:one}
\begin{tabular}{llp{8cm}llll}
  \hline
No & Date & Topic & John & Ali & Ann & Mon \\ 
  \hline
1.00 & 2019-02-05 & Team Formation Formation Formation Formation Formation  & yes & yes & yes & yes \\ 
  2.00 & 2019-02-06 & Team Formation & yes & yes & yes & yes \\ 
  3.00 & 2019-02-07 & Team Formation & yes & yes & yes & yes \\ 
  4.00 & 2019-02-08 & Team Formation & yes & yes & yes & yes \\ 
  5.00 & 2019-02-09 & Team Formation & yes & yes & yes & yes \\ 
  6.00 & 2019-02-10 & Team Formation & yes & yes & yes & yes \\ 
  7.00 & 2019-02-11 & Team Formation & yes & yes & yes & yes \\ 
  8.00 & 2019-02-12 & Final Meeting & yes & yes & yes & yes \\ 
   \hline
\end{tabular}
\end{table}





\subsection{Peer-assessment}

%Edit here 
\blindtext[2] \\

Same as we did with Table~\ref{tab:one}, we can also generate the peer-assessment table providing that we record things in an excel sheet. 

% latex table generated in R 3.3.0 by xtable 1.8-2 package
% Tue Apr 28 23:37:24 2020
\begin{table}[ht]
\centering
\caption{Peer Assessment out of 100} 
\label{tab:two}
\begin{tabular}{lllll}
  \hline
Peer.Review & John & Ali & Ann & Mon \\ 
  \hline
John & 100 & 100 & 100 & 100 \\ 
  Ali & 100 & 100 & 100 & 100 \\ 
  Ann & 100 & 100 & 100 & 100 \\ 
  Mon & 100 & 100 & 100 & 100 \\ 
   \hline
\end{tabular}
\end{table}




\section*{References}\label{pubs}
\printbibliography[heading =none]

\end{document}
